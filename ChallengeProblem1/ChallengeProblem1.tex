\documentclass[journal,12pt,twocolumn]{IEEEtran}

\usepackage{setspace}
\usepackage{gensymb}

\singlespacing


\usepackage[cmex10]{amsmath}

\usepackage{amsthm}

\usepackage{mathrsfs}
\usepackage{txfonts}
\usepackage{stfloats}
\usepackage{bm}
\usepackage{cite}
\usepackage{cases}
\usepackage{subfig}

\usepackage{longtable}
\usepackage{multirow}

\usepackage{enumitem}
\usepackage{mathtools}
\usepackage{steinmetz}
\usepackage{tikz}
\usepackage{circuitikz}
\usepackage{verbatim}
\usepackage{tfrupee}
\usepackage[breaklinks=true]{hyperref}
\usepackage{graphicx}
\usepackage{tkz-euclide}
\usepackage{float}

\usetikzlibrary{calc,math}
\usepackage{listings}
    \usepackage{color}                                            %%
    \usepackage{array}                                            %%
    \usepackage{longtable}                                        %%
    \usepackage{calc}                                             %%
    \usepackage{multirow}                                         %%
    \usepackage{hhline}                                           %%
    \usepackage{ifthen}                                           %%
    \usepackage{lscape}     
\usepackage{multicol}
\usepackage{chngcntr}

\DeclareMathOperator*{\Res}{Res}

\renewcommand\thesection{\arabic{section}}
\renewcommand\thesubsection{\thesection.\arabic{subsection}}
\renewcommand\thesubsubsection{\thesubsection.\arabic{subsubsection}}

\renewcommand\thesectiondis{\arabic{section}}
\renewcommand\thesubsectiondis{\thesectiondis.\arabic{subsection}}
\renewcommand\thesubsubsectiondis{\thesubsectiondis.\arabic{subsubsection}}


\hyphenation{op-tical net-works semi-conduc-tor}
\def\inputGnumericTable{}                                 %%

\lstset{
%language=C,
frame=single, 
breaklines=true,
columns=fullflexible
}
\begin{document}


\newtheorem{theorem}{Theorem}[section]
\newtheorem{problem}{Problem}
\newtheorem{proposition}{Proposition}[section]
\newtheorem{lemma}{Lemma}[section]
\newtheorem{corollary}[theorem]{Corollary}
\newtheorem{example}{Example}[section]
\newtheorem{definition}[problem]{Definition}

\newcommand{\BEQA}{\begin{eqnarray}}
\newcommand{\EEQA}{\end{eqnarray}}
\newcommand{\define}{\stackrel{\triangle}{=}}
\bibliographystyle{IEEEtran}
\providecommand{\mbf}{\mathbf}
\providecommand{\pr}[1]{\ensuremath{\Pr\left(#1\right)}}
\providecommand{\qfunc}[1]{\ensuremath{Q\left(#1\right)}}
\providecommand{\sbrak}[1]{\ensuremath{{}\left[#1\right]}}
\providecommand{\lsbrak}[1]{\ensuremath{{}\left[#1\right.}}
\providecommand{\rsbrak}[1]{\ensuremath{{}\left.#1\right]}}
\providecommand{\brak}[1]{\ensuremath{\left(#1\right)}}
\providecommand{\lbrak}[1]{\ensuremath{\left(#1\right.}}
\providecommand{\rbrak}[1]{\ensuremath{\left.#1\right)}}
\providecommand{\cbrak}[1]{\ensuremath{\left\{#1\right\}}}
\providecommand{\lcbrak}[1]{\ensuremath{\left\{#1\right.}}
\providecommand{\rcbrak}[1]{\ensuremath{\left.#1\right\}}}
\theoremstyle{remark}
\newtheorem{rem}{Remark}
\newcommand{\sgn}{\mathop{\mathrm{sgn}}}
\providecommand{\abs}[1]{\left\vert#1\right\vert}
\providecommand{\res}[1]{\Res\displaylimits_{#1}} 
\providecommand{\norm}[1]{\left\lVert#1\right\rVert}
%\providecommand{\norm}[1]{\lVert#1\rVert}
\providecommand{\mtx}[1]{\mathbf{#1}}
\providecommand{\mean}[1]{E\left[ #1 \right]}
\providecommand{\fourier}{\overset{\mathcal{F}}{ \rightleftharpoons}}
%\providecommand{\hilbert}{\overset{\mathcal{H}}{ \rightleftharpoons}}
\providecommand{\system}{\overset{\mathcal{H}}{ \longleftrightarrow}}
	%\newcommand{\solution}[2]{\textbf{Solution:}{#1}}
\newcommand{\solution}{\noindent \textbf{Solution: }}
\newcommand{\cosec}{\,\text{cosec}\,}
\providecommand{\dec}[2]{\ensuremath{\overset{#1}{\underset{#2}{\gtrless}}}}
\newcommand{\myvec}[1]{\ensuremath{\begin{pmatrix}#1\end{pmatrix}}}
\newcommand{\mydet}[1]{\ensuremath{\begin{vmatrix}#1\end{vmatrix}}}
\numberwithin{equation}{subsection}
\makeatletter
\@addtoreset{figure}{problem}
\makeatother
\let\StandardTheFigure\thefigure
\let\vec\mathbf
\renewcommand{\thefigure}{\theproblem}
\def\putbox#1#2#3{\makebox[0in][l]{\makebox[#1][l]{}\raisebox{\baselineskip}[0in][0in]{\raisebox{#2}[0in][0in]{#3}}}}
     \def\rightbox#1{\makebox[0in][r]{#1}}
     \def\centbox#1{\makebox[0in]{#1}}
     \def\topbox#1{\raisebox{-\baselineskip}[0in][0in]{#1}}
     \def\midbox#1{\raisebox{-0.5\baselineskip}[0in][0in]{#1}}
\vspace{3cm}
\title{Challenge Problem 1}
\author{K.A. Raja Babu}
\maketitle
\newpage
\bigskip
\renewcommand{\thefigure}{\theenumi}
\renewcommand{\thetable}{\theenumi}
Download latex-tikz code from 
%
\begin{lstlisting}
https://github.com/ka-raja-babu/Matrix-Theory/tree/main/ChallengeProblem1
\end{lstlisting}
%
\section{Challenge Question 1}
Show that the matrix $(t\vec{I}-\vec{n}\vec{n}^T)$ in the given document is a rank 1 matrix for a parabola.
%
\section{Solution}
Let
\begin{align}
    \vec{V} = (t\vec{I}-\vec{n}\vec{n}^T) 
\end{align}
where,
\begin{align}
    t &= \frac{\norm{\vec{n}}^2}{e^2}
\end{align}

Let rank of matrix be represented by $\rho$.

\begin{theorem}
\label{theorem1}
A matrix $\vec{A}$ is orthogonally diagonizable if and only if $\vec{A}$ is symmetric.
\end{theorem}

\begin{proof}
Let us assume that $\vec{A}$ is orthogonally diagonizable such that 
\begin{align}
    \vec{D} &= \vec{P}^{-1}\vec{A}\vec{P}
    \\
    \vec{P}^T &= \vec{P}^{-1}
\end{align}
Now,
\begin{align}
    \vec{A} &= \vec{P}\vec{D}\vec{P}^{-1}
    \\
    \implies \vec{A} &= \vec{P}\vec{D}\vec{P}^T \label{eqr1}
\end{align}
Now,
\begin{align}
    \vec{A}^T &= (\vec{P}\vec{D}\vec{P}^T)^T
    \\
    \implies \vec{A}^T &= (\vec{P}^T)^T\vec{D}^T\vec{P}^T
    \\
    \implies \vec{A}^T &= \vec{P}\vec{D}\vec{P}^T \label{eqr2}
\end{align}
$\therefore$ From \eqref{eqr1} and \eqref{eqr2},
\begin{align}
    \vec{A} = \vec{A}^T
\end{align}
Hence,$\vec{A}$ is orthogonally diagonizable only when $\vec{A}$ is symmetric.
\end{proof}

\begin{lemma}
\label{lemma1}
Let $\vec{A}$ be a mxn diagonizable matrix.Then,
\begin{align}
   \rho(\vec{A}) = \rho(\vec{B}^{-1}\vec{A}\vec{B}) = \rho(\vec{D}) \label{eqq1}
\end{align}
\end{lemma}

\begin{proof}
Here , $\vec{A}$ and $\vec{D}$ are similar matrices such that 
\begin{align}
    \vec{D} = \vec{B}^{-1}\vec{A}\vec{B}
\end{align}
Now,
\begin{align}
    \vec{B}\vec{D} &= (\vec{B}\vec{B}^{-1})\vec{A}\vec{B}
    \\
    \implies \vec{B}\vec{D} &= \vec{I}\vec{A}\vec{B}
    \\
    \implies \vec{B}\vec{D} &= \vec{A}\vec{B}
\end{align}
So,
\begin{align}
    \rho(\vec{A}\vec{B}) = \rho(\vec{B}\vec{D}) \label{eq1}
\end{align}
\\
Since $\vec{B}$ is an invertible matrix and hence a full rank matrix.
\\
So,
\begin{align}
    \rho(\vec{A}\vec{B}) = \rho(\vec{A}) \label{eq2}
    \\
    \rho(\vec{B}\vec{D}) = \rho(\vec{D}) \label{eq3}
\end{align}
$\therefore$ Using \eqref{eq1},\eqref{eq2} and \eqref{eq3},
\begin{align}
    \rho(\vec{A}) = \rho(\vec{D})
\end{align}
\end{proof}

\begin{definition}
Rank of a diagonal matrix is equal to the number of its non-zero eigen values. \label{def1}
\end{definition}

Let trace of matrix be represented by $tr$

Now,
\begin{align}
    tr(\vec{V}) &= tr(t\vec{I}-\vec{n}\vec{n}^T)
    \\
    \implies tr(\vec{V}) &= tr(t\vec{I}) - tr(\vec{n}\vec{n}^T)
    \\
    \implies tr(\vec{V}) &= 2t - \norm{\vec{n}}^2
    \\
    \implies tr(\vec{V}) &= \frac{\norm{\vec{n}}^2(2-e^2)}{e^2} \label{trace}
\end{align}

\begin{lemma}
\label{eigen}
Eigen values of a 2x2 matrix $\vec{A}$ are :
\begin{align}
    \lambda &= \frac{tr(\vec{A}) \pm \sqrt{tr(\vec{A})^{2}-4\abs{\vec{A}}}}{2}
\end{align}
\end{lemma}
\begin{proof}
Let $\vec{A}$ = \myvec{a & b\\  c & d}.
\\
Then,characteristic polynomial $p(\lambda)$ is
\begin{align}
    p(t) &= \abs{\vec{A-\lambda\vec{I}}}
    \\
    &= (a-\lambda)(d-\lambda)-bc
    \\
    &= \lambda^2 -(a+d)\lambda + ad-bc
    \\
    &= \lambda^2 -tr(\vec{A})\lambda + \abs{\vec{V}}
\end{align}

So,eigen values which are roots of $p(\lambda)$ are
\begin{align}
    \lambda &= \frac{tr(\vec{A}) \pm \sqrt{tr(\vec{A})^{2}-4\abs{\vec{A}}}}{2}
\end{align}

\end{proof}

$\therefore$ Using \eqref{trace}, for a parabola where $e$=1,
\begin{align}
    tr(\vec{V}) &=\norm{\vec{n}}^2
\end{align}

$\therefore$ Using lemma \ref{eigen}, eigen values are:
\begin{align}
    \lambda_1 &= 0
    \\
    \lambda_2 &= \norm{\vec{n}}^2
\end{align}

Now,
\begin{align}
    \vec{V}^T &= (\norm{\vec{n}}^2\vec{I}-\vec{n}\vec{n}^T)^T
    \\
    \implies \vec{V}^T &= (\norm{\vec{n}}^2\vec{I})^T-(\vec{n}\vec{n}^T)^T
    \\
    \implies \vec{V}^T &= \norm{\vec{n}}^2(\vec{I}^T) - ((\vec{n}^T)^T\vec{n}^T)
    \\
    \implies \vec{V}^T &= (\norm{\vec{n}}^2\vec{I}-\vec{n}\vec{n}^T)
    \\
    \implies \vec{V}^T &= \vec{V}    \label{symm}
\end{align}
So,using \eqref{symm},$\vec{V}$ is a symmetric matrix .


$\therefore$ Using \eqref{symm} and theorem \ref{theorem1}, $\vec{V}$ is a diagonizable matrix such that
\begin{align}
    \vec{P}^{-1}\vec{V}\vec{P} &= \vec{D} 
    \\
    \vec{D} &= \myvec{0 & 0 \\ 0 & \norm{\vec{n}}^2} \label{diag}
\end{align}

$\therefore$
Using def.\ref{def1},lemma \ref{lemma1} and \eqref{diag}.
\begin{align}
    \rho(\vec{V}) = \rho(\vec{D}) = 1
\end{align}

\end{document}

